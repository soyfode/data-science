\chapter{Álgebra lineal}

\section{Matrices}

%----------definición 1.1.
\begin{tcolorbox}[colframe=white]
    \begin{def.}[Matriz]
	Con $m,n \in \mathbb{N}$, una matriz $A$ de valor real $(m, n)$ es una $m \cdot n-tupla$ de elementos $a_ij$, $i = 1,. . . , m$, $j = 1,. . . , n$, que se ordena de acuerdo con un esquema rectangular que consta de $m$ filas y $n$ columnas:	
	\begin{center}
	    $ A = \begin{bmatrix}
		a_{11} & a_{12} & ... & a_{in}\\
		a_{21} & a_{22} & ... & a_{2n}\\
		. & . &  & . \\ 
		. & . &  & . \\ 
		. & . &  & . \\ 
		a_{m1} & a_{m2} & ... % a_{mn}\\
	    \end{bmatrix}, \quad a_{ij} \in \mathbb{R}$
	\end{center}
    \end{def.}
\end{tcolorbox}

\subsection{Suma y multiplicación de matrices}

%----------definición 1.2
\begin{tcolorbox}[colframe=white]
    \begin{def.}
	La suma de dos matrices $A \in \mathbb{R}^{m\cdot n}, \; B \in \mathbb{R}^{m\cdot n}$ se define como la suma de elementos, es decir,
	\begin{center}
	    $ A +B := \begin{bmatrix}
		a_{11}+b_{11}&...&a_{1n} + b_{1n}\\
		.&&.\\
		.&&.\\
		.&&.\\
		a_{m1} + b_{m1}& ... & a_{mn} + b_{mn}\\
	    \end{bmatrix} \in \mathbb{R}^{m  \cdot n}$
	\end{center}
    \end{def.}
\end{tcolorbox}

%----------definición 1.3
\begin{tcolorbox}[colframe=white]
    \begin{def.} Para las matrices $A\in \mathbb{R}^{m\cdot n}$, $B\in \mathbb{R}^{n\cdot k}$, los elementos $c_{ij}$ del producto $C=AB \in \mathbb{R}^{m\cdot k}$ son calculados como:
	$$c_{ij} = \sum\limits_{l=1}^{n} a_{il} b_{lj}, \qquad i=1,...,m \qquad j=1,...,k.$$
    \end{def.}
\end{tcolorbox}

%----------definición 1.4
\begin{tcolorbox}[colframe=white]
    \begin{def.}[Matriz Identidad] En $\mathbb{R}^{n\cdot n},$ definimos la matriz identidad como:
	$$I_n:= \begin{bmatrix}	
	    1&0&...&0&...&0\\
	    0&1&...&0&...&0\\
	    \vdots&\vdots&\ddots&\vdots&\ddots&\vdots\\
	    0&0&...&1&...&0\\
	    \vdots&\vdots&\ddots&\vdots&\vdots&\ddots\\
	    0&0&...&0&...&1\\
	\end{bmatrix} \in \mathbb{n\cdot n}$$
    \end{def.}
\end{tcolorbox}

\begin{tcolorbox}[colframe=white]
    \begin{itemize}
	\item Asociativa: $$\forall A \in \mathbb{R}^{m\cdot n}, B \in \mathbb{R}^{n\cdot p}, C \in \mathbb{R}^{p\cdot q}: (AB)C=A(BC)$$
	\item Distributiva: $$\forall A,B \in \mathbb{R}^{m\cdot n}, C,D \in \mathbb{n\cdot p}: (A+B)C = AC + BC \qquad A(C+D)=AC + AD$$
	\item Multiplicación con la matriz identidad: $$\forall A \in \mathbb{R}^{m\cdot n}: I_m A = AI_n = A$$ Note que $I_m \neq I_n$ para $m\neq n$
    \end{itemize}
\end{tcolorbox}
